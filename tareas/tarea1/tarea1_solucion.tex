\documentclass[letterpaper,11pt]{article}
% Soporte para los acentos.
\usepackage[utf8x]{inputenc}
\usepackage[T1]{fontenc}    
% Idioma español.
\usepackage[spanish,mexico, es-tabla]{babel}
% Soporte de símbolos adicionales (matemáticas)
\usepackage{multirow}
\usepackage{amsmath}		
\usepackage{amssymb}		
\usepackage{amsthm}
\usepackage{amsfonts}
\usepackage{latexsym}
\usepackage{enumerate}
\usepackage{ragged2e}
% Modificamos los márgenes del documento.
\usepackage[lmargin=2cm,rmargin=2cm,top=2cm,bottom=2cm]{geometry}

\title{Universidad Nacional Autónoma de México \\
       Facultad de Ciencias \\
       Estructuras Discretas \\ 
       Tarea 1}
\author{Rubí Rojas Tania Michelle \\
        taniarubi@ciencias.unam.mx \\
        \# cuenta: 315121719}
\date{1 de septiembre de 2017}

\begin{document}
\maketitle

\begin{enumerate}
    % Ejercicio 1.
    \item Demuestre que las siguientes expresiones están bien formadas.    
    
    \begin{itemize}
        % Ejercicio 1.1
        \item $- ((a + b) * c) + 1$
        % Ejercicio 1.2
        \item $((p → q) \land (r → s)) \lor r$
    \end{itemize}

    % Ejericicio 2.
    \item Determine cuáles de las siguientes oraciones son proposiciones 
    atómicas, cuáles son proposiciones no atómicas y cuáles no son 
    proposiciones. Justifique su respuesta. 

    \begin{itemize}
        % Ejercicio 2.1
        \item[a)] El grito de Dolores, en $1810$, sentó las bases para la 
        independencia de México.

        \textsc{Solución:} Esta oración es una proposición ya que puede 
        calificarse como falso o verdadero, y es atómica porque no puede 
        descomponerse en más proposiciones debido a que no contiene conectivos 
        lógicos.

        % Ejercicio 2.2
        \item[b)] Para pasar el examen es necesario que los alumnos estudien, 
        hagan la tarea y asistan a clase.

        \textsc{Solución:} Esta oración es una proposición ya que puede 
        calificarse como falso o verdadero, y es compuesta porque puede 
        descomponerse en más proposiciones debido a que contiene los conectivos
        lógicos \textit{es necesario}, e $y$.

        % Ejercicio 2.3
        \item[c)] $a^{3} + 3a^{2}b + 3ab^{2} + a^{3}$

        \textsc{Solución:} Esta oración no es una proposición ya que no puede 
        calificarse como falso o verdadero.

        % Ejercicio 2.4
        \item[d)] $x \neq y$. (Donde el operador binario $\neq$ evalúa a 
        \textbf{verdadero} si $x$ es distinto de $y$ y a \textbf{falso} si 
        $x$ es igual a $y$)

        \textsc{Solución:} Esta oración es una proposición ya que puede 
        calificase como falso o verdadero (gracias a su operador binario), y es 
        atómica porque no puede descomponerse en más proposiciones debido a que 
        no contiene conectivos lógicos. 

        % Ejercicio 2.5
        \item[e)] Asgard es el mundo de los AEsir y en Svartálfaheim habitan los 
        Svartalfar.

        \textsc{Solución:} Esta oración es una proposción ya que puede 
        calificarse como falso o verdadero, y es compuesta porque contiene el 
        conectivo lógico $y$. 

    \end{itemize}

    % Ejercicio 3.
    \item De los incisos de la pregunta anterior que son proposiciones, exhiba
    una traducción al lenguaje de la lógica proposicional.

    % Ejercicio 4.
    \item Coloque los paréntesis en las siguientes expresiones de acuerdo a la 
    precedencia y asociatividad de los operadores, sin preocuparse por la
    evaluación de la expresión.

    \begin{itemize}
        % Ejercicio 4.1
        \item[a)] $-b + b * * 2 - 4 \cdot a \cdot c / 2 \cdot a$ 
        % Ejercicio 4.2
        \item[b)] $p \land q \lor r → s ↔ p \lor q$
        % Ejercicio 4.3
        \item[c)] $a < b \land b < c → a < b$
        % Ejercicio 4.4
        \item[d)] $a \cdot b - a \cdot c ↔ a > 0 \land b > c$
    \end{itemize}

    % Ejercicio 5.
    \item Ejecute las siguientes sustituciones textuales simultáneas, fijándose
    bien en la colocación de los paréntesis. Quite los paréntesis que son 
    redundantes.

    \begin{itemize}
        % Ejercicio 5.1
        \item[a)] $5x + 3y * a - 4y[y := x]$
        % Ejercicio 5.2
        \item[b)] $(5x + 3y * a - 4y)[y := x]$
        % Ejercicio 5.3
        \item[c)] $(5x + 3y * a - 4y)[y, \; x := x, \;y]$
        % Ejercicio 5.4
        \item[d)] $(5x + 3y * a - 4y)[y := x][x := 3]$
    \end{itemize}

    % Ejercicio 6.
    \item Para las siguientes expresiones, determine a qué esquema pertenecen,
    dé el rango y conectivo principal. Justifique su respuesta.

    \begin{itemize}
        % Ejercicio 6.1
        \item[a)] $((p \land q) \lor (r → s)) → r$
        % Ejercicio 6.2
        \item[b)] $p \lor q → r → s → t$
    \end{itemize}

    % Ejercicio 7.
    \item Para cada una de las expresiones del ejercicio anterior, construya
    los árboles de análisis sintáctico.

    % Ejercicio 8.
    \item Llene las partes que faltan y escriba en qué consiste la expresión 
    $E$.

    % Ejercicio 9.
    \item Utilizando únicamente la tabla de equivalencias dada en clase, 
    demuestre las siguientes equivalencias lógicas mediante razonamiento 
    ecuacional. Justifique cada paso.

    \begin{itemize}
        % Ejercicio 9.1
        \item[a)] $(A \lor B) → Q ≡ (A → Q) \land (B → Q)$ 
        \begin{proof}
            \begin{align*}
                (A \lor B) → Q 
                ≡& \; \neg (A \lor B) \lor Q  
                && \text{equivalencia de $→$} \\
                ≡& \; (\neg A \land \neg B) \lor Q 
                && \text{De Morgan} \\ 
                ≡& \; (\neg A \lor Q) \land (\neg B \lor Q)
                && \text{distributividad} \\ 
                ≡& \; (A → Q) \land (B → Q)
                && \text{equivalencia de $→$} 
            \end{align*}
        \end{proof}

        % Ejercicio 9.2
        \item[b)] $(A \land B) → Q ≡ (A → Q) \lor (B → Q)$ 
        \begin{proof}
            \begin{align*}
                (A \land B) → Q 
                ≡& \; \neg (A \land B) \lor Q
                && \text{equivalencia de $→$} \\ 
                ≡& \; (\neg A \lor \neg B) \lor Q
                && \text{De Morgan} \\
                ≡& \; (\neg A \lor \neg B) \lor (Q \lor Q)
                && \text{idempotencia} \\ 
                ≡& \; (\neg A \lor Q) \lor (\neg B \lor Q)
                && \text{distributividad} \\
                ≡& \; (A → Q) \lor (B → Q)
                && \text{equivalencia de $→$}
            \end{align*}
        \end{proof}

        % Ejercicio 9.3
        \item[c)] $(A \land B) → Q ≡ A → (B → Q)$
        \begin{proof}
            \begin{align*}
                (A \land B) → Q 
                ≡& \; \neg (A \land B) \lor Q
                && \text{equivalencia de $→$} \\ 
                ≡& \; (\neg A \lor \neg B) \lor Q
                && \text{De Morgan} \\ 
                ≡& \; \neg A \lor (\neg B \lor Q)
                && \text{asociatividad} \\ 
                ≡& \; A → (B → Q)
                && \text{equivalencia de $→$}
            \end{align*}
        \end{proof}
    \end{itemize}

    % Ejercicio 10.
    \item Para cada una de las siguientes fórmulas, determine si son o no 
    satisfacibles. En caso de serlo, muestre un modelo para cada una de ellas,
    y en caso de no serlo, demuestre que cada estado evalúa a falso.

    \begin{itemize}
        % Ejercicio 10.1
        \item[a)] $(P \lor Q) \land \neg P \land \neg Q$ 

        \textsc{Solución:} La fórmula es satisfacible. \\ 
        Primero 

        % Ejercicio 10.2
        \item[b)] $(\neg P \lor Q) → ((P \land R) ↔ ((S \land T) → (U \lor P)))$
    \end{itemize}

    % Ejercicio 11. 
    \item Decida si los siguientes conjuntos son satisfacibles. Justifique 
    su respuesta.

    \begin{itemize}
        % Ejercicio 11.1
        \item $\Gamma = \{p → q, \; p \lor r \land s, \; q → t\}$
        % Ejercicio 11.2
        \item $\Gamma = \{p \lor q \lor r, \; \neg (r \lor \neg s), \; s ↔ t, \;
                          p → \neg t, \; q → (p \lor \neg t)\}$        
    \end{itemize}

    % Ejercicio 12.
    \item Para los siguientes argumentos, decida si son correctos y en caso de 
    no serlo dé una interpretación que haga verdaderas a las premisas y falsa 
    a la conclusión.

    \begin{itemize}
        % Ejercicio 12.1
        \item $p → q, \; p \lor r, \; \neg (r \land s), \; /∴ (p → q) → 
               (q \lor \neg s)$
        % Ejercicio 12.2
        \item $p \lor q, \; \neg (p \land r), \; \neg q \; /∴ r → s$
    \end{itemize}

    % Ejercicio 13.
    \item Construya las siguientes derivaciones.

    \begin{itemize}
        % Ejercicio 13.1
        \item $p \land (\neg r \land \neg w), \; l, \; r \land z ⊢ \neg r 
               \land (l \land z)$
        % Ejercicio 13.2
        \item $p \lor \neg(r \lor s), \; r, \; l → \neg p ⊢ \neg l$
        % Ejercicio 13.3
        \item $⊢(p → q) → (p \lor q → q)$
    \end{itemize}

    % Ejercicio 14.
    \item Construya la derivación del siguiente argumento para demostrar que es 
    correcto.

    Si procastinas en Helheim o en Asgard, entonces eres un AEsir. Procastinas 
    en Helheim. Pero, ser gobernado por Odín, es necesario para ser un AEsir. 
    Por lo tanto, eres gobernado por Odín o procastinas en Asgard.

    % Ejercicio 15.
    \item Usando Tableaux, determine la correctud del siguiente argumento.

    \begin{center}
        $(P → Q) → R, \; P, \; R → T \; /∴ T \lor Q$
    \end{center}

    % Ejercicio 16.
    \item Usando Tableaux, demuestre que la siguiente fórmula es una tautología.

    \begin{center}
        $p \lor (\neg p \land q) → p \lor q$
    \end{center}
\end{enumerate}

\end{document}