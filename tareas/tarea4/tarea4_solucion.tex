\documentclass[letterpaper,11pt]{article}
% Soporte para los acentos.
\usepackage[utf8x]{inputenc}
\usepackage[T1]{fontenc}    
% Idioma español.
\usepackage[spanish,mexico, es-tabla]{babel}
% Soporte de símbolos adicionales (matemáticas)
\usepackage{multirow}
\usepackage{amsmath}		
\usepackage{amssymb}		
\usepackage{amsthm}
\usepackage{amsfonts}
\usepackage{latexsym}
\usepackage{enumerate}
\usepackage{ragged2e}
% Modificamos los márgenes del documento.
\usepackage[lmargin=2cm,rmargin=2cm,top=2cm,bottom=2cm]{geometry}

\title{Universidad Nacional Autónoma de México \\
       Facultad de Ciencias \\
       Estructuras Discretas \\ 
       Tarea 4}
\author{Rubí Rojas Tania Michelle \\
        taniarubi@ciencias.unam.mx \\
        \# cuenta: 315121719}
\date{4 de diciembre de 2017}

\begin{document}
\maketitle

\begin{enumerate}
    % Ejercicio 1.
    \item Mostrar que la composición de relaciones es asociativa, es decir, si
    $R, S$ y $T$ son relaciones binarias, entonces

    \begin{center}
        $R ∘ (S ∘ T) = (R ∘ S) ∘ T$
    \end{center}

    \begin{proof} .
        \begin{itemize}
            \item[$⊆$] Supongamos que $(a, b) \in R ∘ (S ∘ T)$. Entonces existe 
            $c \in A$ tal que $(a, c) \in R$ y $(c, b) \in (S ∘ T)$. Por 
            definición, existe $d \in A$ tal que $(c,d) \in S$ y 
            $(d, b) \in T$. Como $(a, c) \in R$ y $(c,d) \in S$, entonces 
            $(a, d) \in R ∘ S$. Además, $(d, b) \in T$ por lo que 
            $(a, b) \in (R ∘ S) ∘ T$.

            \item[$⊇$] Supongamos que $(a,b) \in (R ∘ S) ∘ T$. Entonces, existe
            $c \in A$ tal que $(a,c) \in (R ∘ S)$ y $(c,b) \in T$. Por definición,
            existe $d \in A$ tal que $(a, d) \in R$ y $(d, c) \in S$. Como $
            (d,c) \in S$ y $(c,b) \in T$, entonces tenemos que $(d, b) \in S ∘ T$.
            Además, $(a, d) \in R$, por lo que $(a, b) \in R ∘ (S ∘ T)$.  
        \end{itemize}
    \end{proof}

    % Ejercicio 2.
    \item Sean $A = \{0, 1, 2, 3\}$, $R = \{(a,b) \; | \; a+1\}$ y 
    $S = \{(a,b) \; | \; a = b+2\}$. Realice lo siguiente:

    \begin{itemize}
        % Ejercicio 2.1
        \item Calcule $R$ y $S$

        \textsc{Solución:} $R = \{(0,1), (1,2), (2,3)\}$ y $S = \{(2,0), (3,1)\}$

        % Ejercicio 2.2
        \item Calcule $R ∘ S$

        \textsc{Solución:} $R ∘ S = \{(1,0), (2,1)\}$

        % Ejercicio 2.3
        \item Calcule $R^{3}$

        \textsc{Solución:} $\varnothing$.
    \end{itemize}

    % Ejercicio 3.
    \item Demuestra que $(R^{-1})^{-1} = R$.

    \begin{proof}
        Tenemos que 
        \begin{center}
            $(x, y) \in R ⇔ (y, x) \in R^{-1} ⇔ (x, y) \in (R^{-1})^{-1}$
        \end{center}
    \end{proof}

    % Ejercicio 4.
    \item Para las siguientes relaciones, argumente si cumplen o no con las 
    propiedades de reflexividad, simetría, antisimetría y transitividad.

    \begin{itemize}
        % Ejercicio 4.1
        \item Sean $R$ la relación definida en los números reales por $xRy$
        si y sólo si $x \leq y$.

        \textsc{Solución:}
        \begin{itemize}
            \item Reflexividad. Esta relación cumple con esta propiedad ya que 
            para cualquier $x \in R$ se tiene que $xRx$ pues $x \leq x$.

            \item Simetría. Esta relación no cumple con esta propiedad ya que
            $2, 3 \in R$ y entonces $2 \leq 3$ pero $3 \not \leq 2$.

            \item Antisimetría. Esta relación cumple con esta propiedad ya que 
            para cualquier $x, y \in R$ se tiene que si $xRy \land yRx$ entonces 
            $x \leq y \land y \leq x$ y por la antisimétría de $\leq$, obtenemos 
            que $x = y$.

            \item Transitividad. Esta relación cumple con esta propiedad ya
            que para cualquier $x, y, z \in R$ se tiene que si $xRy \land yRz$
            entonces $x \leq y \land y \leq z$ y por transitividad de $\leq$
            obtenemos que $x \leq z$, lo cual implica que $xRz$. 
        \end{itemize}

        % Ejercicio 4.2
        \item Sean $A = \{a,b,c,d\}$ y $R = \{(a,a), (a,b), (a,c), (b,a), (b,b),
        (b,c), (b,d), (d,d)\}$

        \textsc{Solución:}
        \begin{itemize}
            \item Reflexividad. Esta relación no cumple con esta propiedad ya
            que $(c, c) \not \in R$.

            \item Simetría. Esta relación no cumple con esta propiedad ya que 
            $(a, c) \in R$ pero $(c, a) \not \in R$.

            \item Antisimetría. Esta relación no cumple con esta propiedad ya
            que $(a, b), (b, a) \in R$, pero $a \not = b$.

            \item Transitividad. Esta relación no cumple con esta propiedad ya 
            que $(a, b), (b, d) \in R$, pero $(a, d) \not \in R$.
        \end{itemize}

        % Ejercicio 4.3
        \item Sea $A = ℤ^{+} × ℤ^{+}$ y $R$ la relación $R = \{(a,b), (c,d)) \; 
        | \; a + d = b + c\}$

        \textsc{Solución:}
        \begin{itemize}
            \item Reflexividad. Esta relación cumple con esta propiedad ya que 
            para cualquier $(a, b) \in R$ tenemos que $a + b = b + a$, por lo 
            que $((a, b), (b, a)) \in R$.

            \item Simetría. Esta relación cumple con esta propiedad ya que 
            si $((a, b), (c, d)) \in R$ entonces $a + d = b + c$, y por la 
            conmutatividad de la suma en $ℤ$ tenemos que $c + b = d + a$, 
            lo que implica que $((c, d), (a, b)) \in R$. 

            \item Antisimetría. Esta relación no cumple con esta propiedad 
            ya que \\ $((1, 1), (2, 2)), ((2, 2), (1,1)) \in R$ pero 
            $((1,1) \not = (2,2))$.
            \item Transitividad. Esta relación cumple con esta propiedad ya que 
            si \\ $((a, b), (c, d)), ((c, d), (e, f)) \in R$ entonces 
            $a + d = b + c \land c + f = d + e$. Sumando ambas igualdades 
            tenemos que $a + d + c + f = b + c + d + e$ y usando la ley de la 
            cancelación en la suma obtenemos $a + f = b + e$, lo que implica 
            que $((a, b), (e, f)) \in R$. 
        \end{itemize}
    \end{itemize}

    % Ejercicio 5.
    \item Conteste los siguientes incisos.
    \begin{itemize}
        % Ejercicio 5.1
        \item ¿Puede una relación en un conjunto no ser reflexiva ni
        antirreflexiva? Justifique su respuesta.

        \textsc{Solución:} Sí. Consideremos el conjunto $A = \{1, 2\}$ y la 
        relación $R = \{(1,1)\}$. Como $(2,2) \not \in R$ entonces $R$ no es 
        reflexiva y como $(1, 1) \in R$ entonces $R$ tampoco es antirreflexiva.

        % Ejercicio 5.2
        \item ¿Puede una relación en un conjunto ser simétrica y 
        antisimetríca? ¿O ser asimétrica y antisimétrica también? Justifique su
        respuesta.  
        
        \textsc{Solución:} Sí. Consideremos el conjunto $A = \{1, 2\}$ y las 
        relaciones $R_{1} = \{(1,1)\}$ y $R_{2} = \{(1,2)\}$. Entonces tenemos 
        que $R_{1}$ es simétrica y antisimétríca, mientras que $R_{2}$ es 
        antisimétrica y asimétrica. 
    \end{itemize}

    % Ejercicio 6.
    \item Muestre que una relación $R$ sobre $A$:
    \begin{itemize}
        % Ejercicio 6.1
        \item Es reflexiva si y sólo si $I_{A} ⊆ R$
        % Ejercicio 6.2
        \item Es simétrica si y sólo si $R = R^{-1}$
        % Ejercicio 6.3
        \item Es transitiva si y sólo si $R ∘ R ⊆ R$
    \end{itemize}

    % Ejercicio 7.
    \item Sea $A = ℝ$. Definimos $R ⊆ A × A$ donde $R = \{(x, y) \; | \; 
    ⌊2x⌋ = ⌊2y⌋\}$ donde $⌊2x⌋$ se define como el mayor entero $i \in ℤ$ tal que 
    $i \leq x$.
    \begin{itemize}
        % Ejercicio 7.1
        \item Verifique que $R$ sea una relación de equivalencia.
        % Ejercicio 7.2
        \item Determine las clases de equivalencia de $\frac{1}{4}$ y 
        $\frac{1}{2}$.
        % Ejercicio 7.3
        \item Describa la partición de $ℝ$ en clases de equivalencia.
    \end{itemize}

    % Ejercicio 8.
    \item Determine si las siguientes relaciones son de equivalencia. Si lo son,
    demuestre cada una de las propiedades, si no lo son, exhiba un ejemplo de  
    por qué no se cumple alguna propiedad.
    \begin{itemize}
        % Ejercicio 8.1
        \item $R = \{(a,b) \; : \; a,b \in ℤ, a+b$ es impar$\}$
        % Ejercicio 8.2
        \item $R = \{(a,b) \; : \; a,b \in ℤ, a+b$ es par$\}$
        % Ejercicio 8.3
        \item $R = \{(a,b) \; : \; a,b \in ℤ, |a-b| \leq 5\}$
        % Ejercicio 8.4
        \item $R = \{(a,b) \; : \; a, b \in ℤ, |a-b| < 1\}$
    \end{itemize}

    % Ejercicio 9.
    \item Una partición $P_{1}$ es un refinamiento de la partición $P_{2}$ si 
    cada conjunto $P_{1}$ es subconjunto de algún conjunto en $P_{2}$. Muestre 
    que la partición formada por las clases de congruencia módulo $6$ es un 
    refinamiento de la partición formada por las clases de congruencia módulo 
    $3$.

    % Ejercicio 10.
    \item Sea $A$ un subconjunto de $ℕ$, y sea $\preceq$ la relación sobre $A$
    definida por $a \preceq b$ si y sólo si $b = a^{k}$ para alguna $k \in ℕ$,
    para cada $a,b \in A$. Demuestre que $(A, \preceq)$ es un conjunto 
    parcialmente ordenado. ¿Es $(A, \preceq)$ un conjunto totalmente ordenado?

    % Ejercicio 11.
    \item Sea $A$ un conjunto no vacío, y sea $R$ una relación sobre $A$. La 
    relación $R$ es un cuasi-orden si es reflexiva y transitiva. Suponga que 
    $R$ es un cuasi-orden. Sea $\sim$ la relación sobre $A$ definida por 
    $x \sim y$ si y sólo si $xRy$ y $yRx$ para cualesquiera $x, y \in A$.

    \begin{itemize}
        % Ejercicio 11.1
        \item Demuestre que $\sim$ es una relación de equivalencia.
        % Ejercicio 11.2
        \item Sean $x, y, a, b \in A$. Demuestre que si $xRy$  y $y \sim b$,
        entonces $aRb$.
        % Ejercicio 11.3
        \item Considere el conjunto $A/\sim$ definida por $[x]S[y]$ si y sólo 
        si $xRy$. Demuestre que $(A/\sim, S)$ es un conjunto parcialmente
        ordenado.
    \end{itemize}

    % Ejercicio 12.
    \item Sea $(A, \preceq)$ un conjunto parcialmente ordenado, sea $X$ un 
    conjunto, y sea $h: X → A$ una función inyectiva. Sea $\preceq'$ la 
    relación sobre $X$ definida por $x \preceq' y$ si y sólo si $h(x) 
    \preceq h(y)$, para cada $x, y \in X$. Demuestre que $(X, \preceq')$ es 
    un conjunto parcialmente ordenado.
\end{enumerate}
\end{document}