\documentclass[letterpaper,11pt]{article}
% Soporte para los acentos.
\usepackage[utf8x]{inputenc}
\usepackage[T1]{fontenc}    
% Idioma español.
\usepackage[spanish,mexico, es-tabla]{babel}
% Soporte de símbolos adicionales (matemáticas)
\usepackage{multirow}
\usepackage{amsmath}		
\usepackage{amssymb}		
\usepackage{amsthm}
\usepackage{amsfonts}
\usepackage{latexsym}
\usepackage{enumerate}
\usepackage{ragged2e}
% Soporte para la imágenes.
\usepackage{graphicx}
% Modificamos los márgenes del documento.
\usepackage[lmargin=2cm,rmargin=2cm,top=2cm,bottom=2cm]{geometry}

\title{Universidad Nacional Autónoma de México \\
       Facultad de Ciencias \\
       Estructuras Discretas \\ 
       Tarea 3}
\author{Rubí Rojas Tania Michelle \\
        taniarubi@ciencias.unam.mx \\
        \# cuenta: 315121719}
\date{3 de noviembre de 2017}

\begin{document}
\maketitle

\begin{enumerate}
    % Ejericicio 1.
    \item Demuestre que cada una de las siguientes fórmulas se cumple para cada
    $n \in ℕ$.
    \begin{itemize}
        % Ejercicio 1.1
        \item[a)] $∑_{i=1}^{n}(2i -1)^{3} = n^{2}(2n^{2}-1)$
        \begin{proof}
            Inducción sobre $n$.
            \begin{itemize}
                \item Base de inducción. \\
                n = 1. Este caso se cumple ya que 

                $∑_{i=1}^{1}(2i - 1)^{3} = (2(1) - 1)^{3} = (1)^{3} = 1 = 
                1 (2(1) - 1) =(1)^{2}(2(1)^{2} - 1)$

                \item Hipótesis de inducción. \\
                Supongamos que el resultado es cierto para $n \geq 1$, es decir, 
                supongamos que se cumple $\sum_{i=1}^{n}(2i-1)^{3} = 
                n^{2}(2n^{2}-1)$.

                \item Paso inductivo. \\ 
                Tenemos que demostrar que la fórmula es válida para $n + 1$, es 
                decir, que se cumple $\sum_{i=1}^{n+1} (2i - 1)^{3} = 
                (n +1)^{2}(2(n + 1)^{2} - 1) = (n+1)^{2}(2n^{2} + 4n + 1)$. 

                Entonces 
                \begin{align*}
                    \sum_{i=1}^{n+1} 
                    &= \bigg (\sum_{i=1}^{n}(2i-1)^{3} \bigg) + (2(n+1) - 1)^{3} \\ 
                    &= (n^{2}(2n^{2} - 1)) + (2(n+1) - 1)^{3} 
                    && \text{por H.I.} \\ 
                    &= (2^{4} - n^{2}) + (8n^{3} + 12n^{2} + 6n + 1)
                    && \text{desarrolando términos} \\
                    &= 2^{4} + 8n^{3} + 11n^{2} + 6n +1
                    && \text{agrupando términos semejantes} \\ 
                    &= (n+1)^{2}(2n^{2} + 4n + 1)
                    && \text{factorizando}
                \end{align*}
            \end{itemize}
        \end{proof}

        % Ejercicio 1.2
        \item[b)] $∑_{i=0}^{n}\frac{i}{2^{i}} = 2 - \frac{n + 2}{2^{n}}$
        \begin{proof}
            Inducción sobre $n$. 

            \begin{itemize}
                \item Base de inducción. \\ 
                $n = 0$ Este caso se cumple ya que 

                $∑_{i=0}^{0} \frac{0}{2^{0}} = \frac{0}{1} = 0 =
                2 - 2 = 2 - \frac{2}{1}= 2 - \frac{0+2}{2^{0}}$

                \item Hipótesis de inducción. Supongamos que el resultado es 
                válido para $n \geq 0$, es decir, supongamos que se cumple 
                $∑_{i=0}^{n}\frac{i}{2^{i}} = 2 - \frac{n + 2}{2^{n}}$.

                \item Paso inductivo. \\ 
                Tenemos que demostrar que la fórmula es válida para $n + 1$, es 
                decir, que se cumple $∑_{i=0}^{n+1} \frac{i}{2^{i}} = 
                2 - \frac{(n+1) + 2}{2^{n+1}} = 2 - \frac{n+3}{2^{n+1}}$

                Entonces 
                \begin{align*}
                    ∑_{i=0}^{n+1} \frac{i}{2^{i}}
                    &= \bigg (∑_{i=0}^{n}\frac{i}{2^{i}} \bigg ) + 
                    \bigg (\frac{n+1}{2^{n+1}} \bigg) \\ 
                    &= \bigg (2 - \frac{n+2}{2^{n}} \bigg) + 
                    \bigg (\frac{n+1}{2^{n+1}} \bigg) 
                    && \text{por H.I.} \\ 
                    &= 2 - \bigg(\frac{n+2}{2^{n}} + \frac{n+1}{2^{n+1}} \bigg)
                    && \text{asociatividad} \\ 
                    &= 2 - \bigg( \frac{2(n+2) - (n+1)}{2^{n+1}} \bigg)
                    && \text{resolviendo suma} \\ 
                    &= 2 - \bigg(\frac{2n + 4 - n + 1}{2^{n+1}} \bigg)
                    && \text{simplificando} \\ 
                    &= 2 - \bigg( \frac{n - 3}{2^{n+1}}\bigg)
                    && \text{simplificando}
                \end{align*}
            \end{itemize}
        \end{proof}
    \end{itemize}

    % Ejercicio 2.
    \item Demuestre cada una de las siguientes desigualdades para los valores
    de $n \in ℕ$ especificados.
    \begin{itemize}
        % Ejercicio 2.1
        \item[a)] $(1 + \frac{1}{n})^{n} < n$ para cada $n \in ℕ$ tal que 
        $n \geq 3$

        \begin{proof}
            Inducción sobre $n$.
            \begin{itemize}
                \item Base de inducción. \\
                $n = 3$. Este caso se cumple ya que 
                $(1 + \frac{1}{3})^{3} = (\frac{4}{3})^{3} = \frac{64}{27} < 3$.

                \item Hipótesis de inducción. Supongamos que el resultado es 
                válido para $n \geq 3$, es decir, supongamos que se cumple 
                $(1 + \frac{1}{n})^{n} < n$.

                \item Paso inductivo. \\ 
                Tenemos que demostrar que la fórmula se cumple para $n + 1$, es 
                decir, que se cumple $(1 + \frac{1}{n+1})^{n+1} < n+1$.
                
                Sabemos que, por propiedades en los números reales, se cumple 
                $n < n+1 ⇒ \frac{1}{n+1} < \frac{1}{n}$, y por lo tanto, 
                $1 + \frac{1}{n+1} < 1 + \frac{1}{n}$. Además, si $a$ y $b$
                son reales, se tiene que $a < b ⇒ a^{n} < b^{n}, \; n \in ℕ$.
                Podemos utilizar este resultado con la desigualdad 
                $1 + \frac{1}{n+1} < 1 + \frac{1}{n}$, y desarrollar para  
                utilizar nuestra hipótesis.
                
                Entonces 
                \begin{align*}
                    \bigg(1 + \frac{1}{n + 1}\bigg )^{n+1} 
                    &< \bigg (1 + \frac{1}{n} \bigg)^{n+1}
                    && \text{por la observación anterior} \\ 
                    &= \bigg (1 + \frac{1}{n} \bigg)^{n} 
                    \bigg(1 + \frac{1}{n} \bigg) 
                    && \text{descomponiendo la expresión anterior} \\
                    &< n (1 + \frac{1}{n})
                    && \text{por H.I.} \\ 
                    &= n + 1
                    && \text{simplificando}
                \end{align*}
            \end{itemize}
        \end{proof}

        % Ejercicio 2.2
        \item[b)] $7n < 2^{n}$ para cada $n \in ℕ$ tal que $n \geq 6$
        \begin{proof}
            Inducción sobre $n$.

            \begin{itemize}
                \item Base de inducción. \\ 
                $n = 6$. Este caso se cumple ya que $7(6) = 42 < 64 = 2^{6}$.

                \item Hipótesis de inducción. Supongamos que el resultado es 
                válido para $n \geq 6$, es decir, supongamos que se cumple 
                $7n < 2^{n}$.

                \item Paso inductivo. \\ 
                Tenemos que demostrar que la fórmula es válida para $n + 1$, es 
                decir, que se cumple $7(n+1) < 2^{n+1}$.

                Entonces 
                \begin{align*}
                    7(n + 1)
                    &= 7n + 7
                    && \text{simplificando} \\ 
                    &< 2^{n} + 7
                    && \text{por H.I.} \\ 
                    &< 2^{n} + 2^{n}
                    && \text{ya que $7 < 2^{n}$ con $n \geq 6$} \\ 
                    &= 2 \cdot 2^{n}
                    && \text{simplificando} \\ 
                    &= 2^{n+1}
                    && \text{simplificando}
                \end{align*}
            \end{itemize}
        \end{proof}
    \end{itemize}

    % Ejercicio 3.
    \item Demuestre que 
    \begin{center}
        $∏_{i=2}^{n}(1 - \frac{1}{i^{2}}) = (1 - \frac{1}{2^{2}}) \times \cdots
        \times (1 - \frac{1}{n^{2}}) = \frac{n + 1}{2n}$
    \end{center}

    Para cada $n \in ℕ$ tal que $n \geq 2$.

    \begin{proof}
        Inducción sobre $n$.

        \begin{itemize}
            \item Base de inducción. \\ 
            $n = 2$. Este caso se cumple ya que $∏_{i=2}^{2}(1 - \frac{1}{i^{2}})
            = 1 - \frac{1}{2^{2}} = (1 - \frac{1}{4}) = \frac{3}{4} = 
            \frac{2+1}{2(2)}$

            \item Hipótesis de inducción. Supongamos que el resultado es válido 
            para $n \geq 2$, es decir, supongamos que se cumple $∏_{i=2}^{n}
            (1 - \frac{1}{i^{2}}) = \frac{n+1}{2n}$

            \item Paso inductivo. \\ 
            Tenemos que demostrar que la fórmula es válida para $n + 1$, es 
            decir, que se cumple $∏_{i=2}^{n+1} = \frac{(n+1)+1}{2(n+1)} =
            \frac{n+2}{2n+2}$.

            Entonces 
            \begin{align*}
                ∏_{i=2}^{n+1} 
                &= \bigg(∏_{i=2}^{n} \bigg(1-\frac{1}{i^{2}}\bigg) \bigg) ⋅
                \bigg(1 - \frac{1}{(n+1)^{2}}\bigg) \\
                &= \bigg(\frac{n+1}{2n}\bigg) ⋅ \bigg(1 - \frac{1}{(n+1)^{2}}\bigg)
                && \text{por H.I.} \\ 
                &= \bigg(\frac{n+1}{2n}\bigg) ⋅ 
                \bigg(\frac{n^{2}+2n}{(n+1)^{2}}\bigg)
                && \text{resolviendo resta} \\ 
                &= \frac{(n+1)(n^{2}+2n)}{2n(n+1)^{2}}
                && \text{resolviendo multiplicación} \\ 
                &= \frac{n^{2}+2n}{2n(n+1)}
                && \text{eliminando el término $(n+1)$} \\
                &= \frac{n(n+2)}{2n(n+1)}
                && \text{factorizando} \\ 
                &= \frac{n+2}{2n+2}
                && \text{eliminando el término $n$ y simplificando}
            \end{align*}
        \end{itemize}
    \end{proof}

    % Ejercicio 4.
    \item Sean $\{r_{i}\}_{i \in ℕ^{\times}}$ la sucesión definida por 
    $r_{1} = 1$, y $r_{n + 1} = 4r_{n} + 7$ para cada $n \in ℕ$. 
    Demuestre que $r_{n} = \frac{1}{3}(10 \cdot 4^{n - 1} - 7)$ para cada 
    $n \in ℕ^{\times}$. 

    \begin{proof}
        Inducción sobre $n$.

        \begin{itemize}
            \item Base de inducción. \\ 
            $n = 1$. Este caso se cumple ya que $r_{i} = 1 = \frac{1}{3}(3) =
            \frac{1}{3}(10 ⋅ 1 - 7) = \frac{1}{3}(10 ⋅ 4^{0} - 7) = 
            \frac{1}{3}(10 ⋅ 4 ^{1-1} - 7)$

            \item Hipótesis de inducción. Supongamos que el resultado es válido 
            para $n + 1$, es decir, supongamos que se cumple 
            $r_{n} = \frac{1}{3}(10 ⋅ 4^{n - 1} - 7)$.

            \item Paso inductivo. \\ 
            Tenemos que demostrar que la fórmula es válida para $n + 1$, es
            decir, que se cumple $r_{r+1} = \frac{1}{3}(10 ⋅ 4^{n} - 7)$.
            
            Entonces 
            \begin{align*}
                r_{n+1} 
                &= 4r_{n} + 7
                && \text{definición recursiva de $r_{n+1}$} \\ 
                &= 4(\frac{1}{3}(10 ⋅ 4^{n-1} - 7)) + 7
                && \text{definición recursiva de $r_{n}$} \\ 
                &= \frac{1}{3}(10 ⋅ 4^{n-1} ⋅ 4 - 7 ⋅ 4) + 7
                && \text{conmutatividad y simplificando} \\ 
                &= \frac{1}{3}(10 ⋅ 4^{n} - 7 (1+3)) + 7
                && \text{simplificando y aplicando $4 = 3 + 1$} \\
                &= \frac{1}{3}(10 ⋅ 4^{n} - 7 + 21) + 7 
                && \text{simplificando} \\
                &= \frac{1}{3}(10 ⋅ 4^{n} - 7) - \frac{1}{3} ⋅ 21 + 7
                && \text{sacando al $21$} \\ 
                &= \frac{1}{3}(10 ⋅ 4^{n} - 7) - 7 + 7
                && \text{resolviendo multiplicación} \\ 
                &= \frac{1}{3}(10 ⋅ 4^{n} - 7)
                && \text{simplificando}
            \end{align*}
        \end{itemize}
    \end{proof}

    % Ejercicio 5.
    \item Sean $\{b_{i}\}_{i \in ℕ}$ la sucesión definida por $d_{0} = 2, 
    d_{1} = 3, $ y $d_{n} = d_{n-1} \cdot d_{n-2}$ para cada $n \in ℕ$ tal que 
    $n \geq 3$. Encuentre una fórmula explícita para $d_{n}$, y demuestre por 
    inducción que su fórmula funciona.

    \begin{proof}
        La fórmula explícita propuesta es $d_{n} = 2^{F_{n-1}} ⋅ 3^{F_{n}}$, con 
        $n \geq 1$ y donde $F_{n}$ es el $n$-ésimo número de Fibonacci.

        Demostraremos que la fórmula es válida utilizando inducción fuerte sobre 
        $n$.
        \begin{itemize}
            \item Base de inducción. \\
            $n = 1$. Este caso se cumple ya que $d_{1} = 3 = 1 ⋅ 3 = 
            2^{0} ⋅ 3^{1} = 2^{F_{0}} ⋅ 3 ^{F_{1}} = 2^{F_{1-1}} ⋅ 3^{F_{1}}$ 

            \item Hipótesis de inducción. Supongamos que el resultado es válido 
            para $n$, es decir, supongamos que se cumple para 
            $d_{n} = 2^{F_{n-1}} ⋅ 3^{F_{n}}$.

            \item Paso inductivo. \\
            Tenemos que demostrar que la fórmula es válida para $n + 1$, es 
            decir, que se cumple $d_{n+1} = 2^{F_{n}} ⋅ 3^{F_{n+1}}$.

            Entonces 
            \begin{align*}
                d_{n+1} 
                &= d_{(n+1)-1} ⋅ d_{(n+1)-2}
                && \text{definición recursiva de $d_{n}$} \\ 
                &= d_{n} ⋅ d_{n-1}
                && \text{simplificando} \\ 
                &= 2^{F_{n-1}} ⋅ 3^{F_{n}} ⋅ 2^{F_{(n-1)-1}} ⋅ 3^{F_{n-1}}
                && \text{por H.I.} \\ 
                &= 2^{F_{n-1}} ⋅ 2^{F_{n-2}} ⋅ 3^{F_{n}} ⋅ 3^{F_{n-1}}
                && \text{simplificando y aplicando conmutatividad} \\ 
                &= 2^{F_{n-1} + F_{n-2}} ⋅ 3^{F_{n} + F_{n-1}}
                && \text{simplificando} \\ 
                &= 2^{F_{n}} ⋅ 3^{F_{n+1}}
                && \text{definición de $F_{n}$ y $F_{n+1}$}
            \end{align*}
        \end{itemize}
    \end{proof}

    % Ejercicio 6.
    \item Sea \textit{spar(n)} la función definida como $span(n) = 2 + 4 + 6 +
    ⋯ + 2n$. Defina una implementación recursiva llamada $f(n)$ para la función 
    $spar(n)$. Demuestre que $f(n) = n(n + 1)$.

    \begin{proof}
        Definimos la función como $f(0) = 0, f(n+1) = f(n) + 2(n+1)$. 
        Demostraremos, por inducción sobre $n$, que $f(n) = n(n + 1)$.

        \begin{itemize}
            \item Base de inducción. \\ 
            $n = 0$. Este caso se cumple ya que $f(0) = 0 = 0(1) =0(0 + 1)$.

            \item Hipótesis de inducción. Supongamos que el resultado es válido
            para $n$, es decir, supongamos que se cumple $f(n) = n(n + 1)$.

            \item Paso inductivo. \\ 
            Tenemos que demostrar que la fórmula es válida para $n + 1$, es decir,
            que se cumple $f(n + 1) = (n + 1)(n + 2)$.

            Entonces 
            \begin{align*}
                f(n + 1) 
                &= f(n) + 2(n + 1)
                && \text{definición recursiva de $f(n+1)$} \\ 
                &= n(n + 1) + 2(n + 1)
                && \text{por H.I.} \\ 
                &= n^{2} + n + 2n + 2
                && \text{simplificando} \\ 
                &= n^{2} + 3n + 2
                && \text{agrupando términos semejantes} \\ 
                &= (n + 1) (n + 2)
                && \text{factorizando}
            \end{align*}
        \end{itemize}
    \end{proof}

    % Ejercicio 7.
    \item Una cadena de caracteres es palíndroma si es de la forma $ww^{R}$ 
    donde $w^{R}$ es $w$ escrita de atrás hacia adelante, por ejemplo, $0110,
    abbbba, holaaloh$. Defina al conjunto de las cadenas palíndromas 
    recursivamente, y demuestre mediante inducción estructural, que todas las 
    cadenas palíndromas definidas tienen un número par de símbolos. 

    \begin{proof}
        Sea $∑$ el alfabeto sobre el cual construiremos a los palíndromos. 
        Definimos al conjunto $P$ de las cadenas palíndromas de la siguiente 
        forma

        \begin{itemize}
            \item[i)] Para cada $a \in ∑$,  $aa^{R} \in P$.
            \item[ii)] Si $v$ y $w$ son cadenas tales que $vv^{R}$ y $ww^{R}$
            son elementos de $P$, entonces también $wvv^{R}w^{R}$ lo es.
            \item[iii)] Sólo las cadenas obtenidas con las reglas $i)$ y $ii)$
            son elementos de $P$.
        \end{itemize}

        Ahoram demostraremos que todos los elementos del conjunto $P$ tienen un 
        número par de símbolos, utilizando inducción estructural.

        \begin{itemize}
            \item Base de inducción. \\ 
            Cualquier cadena en $P$ construida con la regla $i)$ es de la forma 
            $aa^{R}$ para algún símbolo en $∑$, por lo que tiene exactamente dos 
            elementos.

            \item Hipótesis de inducción. Supongamos que $uu^{R}$ y $vv^{R}$ son 
            cadenas en $P$, y que ambas tienen un número par de símbolos.

            \item Paso inductivo. \\ 
            Demostraremos que la cadena $uvv^{R}u^{R}$ se puede obtener con la 
            regla $ii)$ a partir de las cadenas $u$ y $v$, que tiene un número 
            par de símbolos. Como el número de símbolos $uvv^{R}u^{R}$ es la 
            suma del número de símbolos en $uu^{R}$ y $vv^{R}$, por la hipótesis
            tenemos que éstas cadenas tienen longitud par; y como la suma de un 
            par con otro par es un par, podemos concluir que el resultado 
            deseado.
        \end{itemize}
    \end{proof}

    % Ejercicio 8.
    \item La función $snoc$ en listas se define como sigue:
    \begin{center}
        $snoc \; \; c[x_{1}, ⋯, x_{n}] = [x_{1}, ⋯, x_{n}, c]$
    \end{center}

    \begin{itemize}
        % Ejercicio 8.1
        \item[a)] De una implementación recursiva para $snoc$.

        \textsc{Solución:} Definimos la función $snoc$ de la siguiente forma:
        \begin{center}
            $snoc \; c \; [] = [c]$

            $snoc \; c \; (a:xs) = (a: snoc \; c \; xs)$
        \end{center}

        % Ejercicio 8.2
        \item[b)] Demuestre, usando la definición recursiva, que:
        \begin{center}
            $snoc \; \; c \; (xs\_ys) = xs\_ (snoc \; \; c \; ys)$
        \end{center} 

        \begin{proof}
            Inducción estructural sobre $xs$.

            \begin{itemize}
                \item Base de inducción. \\ 
                $xs = []$. Este caso se cumple ya que 
                $snoc \; c \; ([]\_ys) = snoc \; c \; ys = 
                [] \_ (snoc \; c \; ys)$

                \item Hipótesis de inducción. Supongamos que se cumple
                $snoc \; \; c \; (xs\_ys) = xs\_ (snoc \; \; c \; ys)$.

                \item Paso inductivo. \\ 
                Debemos demostrar que para cualquier $a$, se cumple 
                que $snoc \; c \; ((a:xs)\_ys) = 
                (a:xs)\_ (snoc \; \; c \; ys)$.

                Entonces 
                \begin{align*}
                    snoc \; c \; ((a:xs)\_ys)
                    &= snoc \; c \; (a:(xs\_ys))
                    && \text{asociatividad de la concatenación} \\
                    &= (a: snoc \; c \; (xs\_ys)) 
                    && \text{definición recursiva de $snoc$} \\ 
                    &= (a: xs\_(snoc \; c \; ys))
                    && \text{por H.I.} \\ 
                    &= (a : xs)\_(snoc \; c \; ys)
                    && \text{asociatividad de la concatenación} 
                \end{align*}
            \end{itemize}
        \end{proof}
    \end{itemize}

    % Ejercicio 9.
    \item Considere la siguiente función misteriosa $mist$
    \begin{center}
        $mist \; [] \; ys = ys$

        $mist \; (x : xs) \;  ys = mist \; xs \; (x : ys)$
    \end{center}

    \begin{itemize}
        % Ejercicio 9.1
        \item[a)] ¿Qué hace la función $mist$?

        \textsc{Solución:} La función $mist$ recibe dos listas $xs$ y $ys$, y 
        concatena la reversa de $xs$ con $ys$.

        \item[b)] Muestre que $rev \; xs = mist \; xs \; []$, con $rev$ la 
        operación reversa sobre cadenas definidas cómo sigue: 
        \begin{center}
            $rev \; [] = []$

            $rev \; (a : xs) = rev \; xs\_[a]$
        \end{center}

        \begin{proof}
            Inducción estructural sobre $xs$.

            \begin{itemize}
                \item Base de inducción. \\ 
                $xs = []$. Este caso se cumple ya que $rev \; [] = [] =
                mist \; [] \; []$.

                \item Hipótesis de inducción. Supongamos que se cumple 
                $rev \; xs = mist \; xs \; []$.

                \item Tenemos que demostrar que para cualquier $a$, se cumple
                $rev \; (a:xs) = mist \; (a:xs) \; []$.

                Entonces 
                \begin{align*}
                    rev \; (a:xs)
                    &= rev \; xs\_[a]
                    && \text{definición recursiva de $rev$} \\
                    &= (mist \; xs \; [])\_[a]
                    && \text{por H.I.} \\ 
                    &= mist \; xs \; (a:[]) 
                    && \text{propiedades de $[]$} \\ 
                    &= mist \; (a: xs) \; []
                    && \text{definición recursiva de $mist$}
                \end{align*}
            \end{itemize}
        \end{proof}
    \end{itemize}

    % Ejercicio 10.
    \item Sea $A$ una fórmula de la lógica proposicional cuyos únicos 
    conectivos son $\land, \lor \neg$. Definimos la fórmula dual de $A$, 
    denotada como $A_{D}$, intercambiando $\land$ por $\lor$, $\lor$ por 
    $\land$ y reemplazando a cada variable $p$ por su negación $\neg p$.
    Por ejemplo, $A = (r \lor q) \land \neg p$, $A_{D} = (\neg r \land \neg q)
    \lor \neg \neg p$.
    \begin{itemize}
        % Ejercicio 10.1
        \item[a)] Defina recursivamente una función dual tal que 
        $dual(A) = A_{D}$.

        \textsc{Solución:} Definimos la función $dual(A)$ de la siguiente forma 
        \begin{itemize}
            \item $dual(p) = \neg p$
            \item $dual(true) = false$
            \item $dual(false) = true$
            \item $dual(¬A) = ¬dual(A)$
            \item $dual(A \land B) = dual(A) \lor dual(B)$
            \item $dual(A \lor B) = dual(A) \land dual(B)$. 
        \end{itemize}

        % Ejercicio 10.2
        \item[b)] Muestre que $¬A ≡ A_{D}$ mediante inducción sobre 
        fórmulas.

        \begin{proof}
            Inducción sobre las fórmulas.

            \begin{itemize}
                \item Base de inducción. \\ 
                $A$ es atómica (no tiene operadores). Este caso se cumple, ya 
                que, por la reflexividad de $≡$, tenemos que 
                \begin{itemize}
                    \item $A = p ⇒ dual(p) = ¬p ≡ ¬p$
                    \item $A = true ⇒ dual(true) = false ≡ false$
                    \item $A = false ⇒ dual(false) = true ≡ true$
                \end{itemize}

                \item Hipótesis de inducción. Supongamos que $¬A ≡ A_{D}$ y 
                $¬B ≡ B_{D}$.

                \item Paso inductivo. \\ 
                Tenemos que 
                \begin{itemize}
                    \item $A = ¬A ⇒ dual(¬A) = ¬dual(A) ≡ ¬¬A$ 
                    \item $A = A \land B ⇒ dual(A \land B) = 
                    dual(A) \lor dual(B) ≡ ¬A \lor ¬B ≡ ¬(A \land B)$
                    \item $A = A \lor B ⇒ dual(A \lor B) = 
                    dual(A) \land dual(B) ≡ ¬A \land ¬B ≡ ¬(A \lor B)$
                \end{itemize}
            \end{itemize}
        \end{proof}
    \end{itemize}

    % Ejercicio 11.
    \item Resuelva los siguientes incisos para árboles binarios.
    \begin{itemize}
        % Ejercicio 11.1
        \item Defina recursivamente una función $hmi(T)$ que devuelve la hoja
        más a la izquierda en un árbol binario.

        \textsc{Solución:} Definimos recursivamente la función $hmi(T)$ de la 
        siguiente forma (suponiendo que nuestro árbol es diferente de $void$):
        \begin{itemize}
            \item $hmi(tree(void, c, void)) = c$
            \item $hmi(tree(T_{1}, c, T_{2})) = hmi(T_{{1}})$
            \item $hmi(tree(T_{1}, c, void)) = hmi(T_{{1}})$
            \item $hmi(tree(void, c, T_{2})) = hmi(T_{2})$
        \end{itemize}

        % Ejercicio 11.2
        \item La distancia entre la raíz $r$ de un árbol binario T hacía algún
        otro nodo $p$ es el número de aristas (líneas) que hay entre ambos nodos
        y la altura o profundidad de un árbol se define como la máxima distancia
        entre la raíz y alguna hoja más $1$. Demuestre que el número máximo de 
        hojas en un árbol de altura $n$ es $2^{n - 1}$.

        \begin{proof}
            Inducción estructural sobre $T$.

            \begin{itemize}
                \item Base de inducción. \\ 
                $T = void$. Este caso se cumple ya que la altura y el número de 
                hojas de $T$ son iguales a $0$. Así, $0 < \frac{1}{2} = 2^{-1} = 
                2^{0-1}$

                \item Hipótesis de inducción. Si $T$ es un árbol de altura 
                $n_{i}$, entonces tiene a lo más $2^{n_{i}-1}$ hojas, donde 
                $i \in \{1,2\}$.

                \item Paso inductivo. \\ 
                Sea $T = tree(T_{1}, c, T_{2})$. Notemos que la altura de $T$ es 
                igual a $1 + max \{n_{1}, n_{2}\}$, por lo que debemos demostrar 
                que el número de hojas de $T$ es a lo más 
                $2^{1 + max\{n_{1}, n_{2}\}} = 2^{max\{n_{1}, n_{2}\}}$.

                Como el número de hojas de $T$ es el número de hojas de $T_{1}$
                más el número de hojas de $T_{2}$, entonces podemos definir 
                recursivamente una función $nh$ que calcule el número de hojas de  
                $T$. Así, 
                \begin{align*}
                    nh(T) 
                    &= nh(T_{1}) + nh(T_{2})
                    && \text{definición recursiva de $nh$} \\ 
                    &\leq 2^{n_{1}-1} + 2^{n_{2}-1}
                    && \text{por H.I.} \\
                    &\leq 2^{max\{n_{1}, n_{2}\}-1} + 2^{max\{n_{1}, n_{2}\}-1}
                    && \text{ya que $n_{i} = max\{n_{1}, n_{2}\}$} \\ 
                    &= 2 ⋅ 2^{max\{n_{1}, n_{2}\}-1}
                    && \text{aritmética} \\ 
                    &= 2^{max\{n_{1}, n_{2}\}-1+1}
                    && \text{leyes de exponentes} \\ 
                    &= 2^{max\{n_{1}, n_{2}\}}
                    && \text{simplificando}
                \end{align*}
            \end{itemize}
        \end{proof}
        % Ejercicio 11.3
        \item De una definición recursiva que devuelva en una lista el recorrido
        post-orden de los árboles binarios. Si se tiene el siguiente árbol T, 
        el resultado del recorrido es el siguiente:
        \begin{center}
            \centerline{\includegraphics[scale=0.7]{recorrido.png}}
        \end{center} 
        
        \textsc{Solución:} Definimos recursivamente la función $post-order$ de
        la siguiente forma 
        \begin{itemize}
            \item $post-order(void) = []$
            \item $post-order(tree(T_{1}, c, T_{2})) = 
            post-order(T_{1})\_post-order(T_{2})\_[c]$
        \end{itemize}
    \end{itemize}
\end{enumerate}
\end{document}
